\documentclass[12pt,a4paper]{article}
\usepackage[T1]{fontenc}
\usepackage[utf8]{inputenc}
\usepackage{fancyhdr}
 \fancyhead[L]{}
\fancyhead[R]{}
\renewcommand{\headrulewidth}{0pt}
\renewcommand{\footrulewidth}{1pt}
\usepackage[french]{babel}
\usepackage{color}
\usepackage{amsmath}
\usepackage {graphicx}
\usepackage{array}
\pagestyle{fancy}
\usepackage{parskip}
\usepackage{geometry}
\geometry{hmargin=2.5cm,vmargin=2.5cm}
\begin{document}
\begin{figure}[t]
 \begin{center}
 \includegraphics[height=1.75cm]{logolemansU.png}
\hfill
 \includegraphics[height=2.2cm]{logo_IC2.png}
 \end{center}
 \end{figure}
\title{\color{blue}Le Mans université \\\color{black} License Informatique 2eme année\\ Module 174UP02 Conduite de projets\\\textbf {Projet Morpion fractal}}
\author{\textit{B. Lucie B. Damien A. Amandine et B. Mathilde}}
\date{\today}
\maketitle
\color{black}
\newpage
\tableofcontents
    \newpage
    \section{Introduction}
    Nous avons décidé de réaliser un Morpion Fractal. Il s'agit d'un jeu à deux joueurs dans lequel le but est d'alligner avant son adversaire 3 symboles identiques. Notre version étant une version fractale chaque cases est subdivisées en jeu de morpion. Ainsi pour gagner il faut remporter plusieurs partie de morpion et remporter 3 cases. L'alignement peut être, horizontal, vertical ou en diagonale. L'issue du jeu peut être la victoire de l'un des deux joueurs ou une partie nulle.
    Lors de la parite, l'endroit où le joeur joue detrmine la position que va jouer l'adversaire. En effet si le premier joueur joue dans une des cases situé au milieu d'un des morpion alors l'adversaire devra jouer dans le morpion central.

    image 

    \section{Analyse}
        toutes les fonction 
    \subsection{morpion.c}
    

    \section{Tests effectués et résultats obtenus}


    \section{Conclusion}
\end{document}